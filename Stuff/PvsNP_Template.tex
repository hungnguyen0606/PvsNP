\documentclass[a4paper 14pt]{extarticle}
\usepackage[utf8]{vietnam}
\usepackage{amsmath}
\usepackage{amsthm}
\usepackage{amsfonts}
\usepackage{amssymb}
\usepackage{float}

\usepackage[margin = 1in]{geometry}
\usepackage{graphicx}
\graphicspath{{images/}}
\begin{document}
	\begin{titlepage}
\begin{center}
	\large{\textbf{ĐẠI HỌC QUỐC GIA THÀNH PHỐ HỒ CHÍ MINH}}\\
	\large{\textbf{TRƯỜNG ĐẠI HỌC KHOA HỌC TỰ NHIÊN}}\\
	\large{\textbf{KHOA CÔNG NGHỆ THÔNG TIN}}\\
	\vfill
	\begin{figure}[H]
		\centerline{\includegraphics[scale = 0.5]{logo}}
	\end{figure}

	\Large{\textbf{Nhập môn phân tích độ phức tạp thuật toán}}\\
	\Large{\textbf{Báo cáo tìm hiểu P vs NP}}\\

\end{center}
	\vfill
\begin{flushright}
	
	\begin{tabular}{l l l}
		GVLT: &Thầy Trần Đan Thư\\
		&\\
		GVTH: &Thầy Nguyễn Đức Thân\\
		&Thầy Trương Toàn Thịnh\\
		&Thầy Nguyễn Vinh Tiệp\\
		&Thầy Nguyễn Sơn Hoàng Quốc\\
		&\\
		Nhóm:&13\\
		Sv: &Nguyễn Phan Mạnh Hùng & 1312727\\
		&Lục Kiến Nghiệp & 1312734\\
		&La Ngọc Thùy An & 1312716\\
		&Nguyễn Phước Đạt & 1312721\\
	\end{tabular}
\end{flushright}


\end{titlepage}

	\pagebreak
	
	
	\section{Lịch sử}
	\section{Tổng quan}
	\begin{itemize}
		\item Nguồn gốc của bài toán
		\item Ý nghĩa bài toán trong thực tiễn và lý thuyết (Nếu giải được thì sao? Ko giải được thì sao?)
		\item Mức độ quan trọng của việc giải bài toán
		\item Tại sao bài toán này vẫn chưa giải được (note: bài toán triệu \$)
	\end{itemize}
	
	\section{Định nghĩa}
	\subsection{Class P}
	
	Lớp P bao gồm các bài toán mà ta có thể đưa ra một lời giải trong thời gian đa thức.\\
	\begin{itemize}
		\item Ví dụ một số bài toán 
	\end{itemize}
	\subsection{Class NP}
	Lớp NP bao gồm các bài toán bài toán quyết định mà lời giải có thể kiểm tra được trong thời gian đa thức.
	\begin{itemize}
		\item Nêu định nghĩa np-complete, np-easy
		\item Tính quan trọng của từng loại.
	\end{itemize}
	
	
	\section{Mối quan hện giữa các lớp bài toán}
	\begin{itemize}
		\item Venn diagram
		\item Ví dụ: reduce from graph coloring to hamilton problem
	\end{itemize}
\end{document}